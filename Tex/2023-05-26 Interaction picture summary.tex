\documentclass{article}
\usepackage[cjk,hangul]{kotex}
\usepackage[fleqn]{amsmath}
\usepackage{physics}
\usepackage{mathrsfs}
\usepackage{amssymb}
\usepackage{microtype}
\newcommand{\RN}[1]{%s
  \textup{\uppercase\expandafter{\romannumeral#1}}%
}
\usepackage{geometry}
\geometry{
 a4paper,
 left=20mm,
 right=20mm,
 top=15mm,
 bottom=20mm
 }
\begin{document}
\title{Interaction picture summary}
\author{S.Sora}
\maketitle
\subsection*{Schrodinger picture}
\begin{align*}
    \text{Wavefunction : } \quad &\psi_s = e^{-i\Hat{H}t}\psi \\
    \text{Operator : } \quad &\Hat{A}_s = \hat{A}
\end{align*}
\subsection*{Heisenberg picture}
\begin{align*}
    \text{Wavefunction : } \quad &\psi_h = \psi \\ 
    \text{Hamiltonian : } \quad &\hat{A}_h(t) = e^{-i\Hat{H}t}Ae^{i\Hat{H}t}
\end{align*}
상호작용 묘사(Interaction picture)에서, 해밀토니안은 계(System) 자체를 서술하는 $H_0$ 과 시간에 무관하게 주변 입자들간의 작용을 나타내는 상호작용 해밀토니안 $H_{int}$으로 나뉘어집니다. 
또한 시간에 따른 파동함수는 계의 시간 연산자가 곱해지며($\psi_{i}(t) = e^{i\hat{H}_0t}\psi(\textbf{r})$), 
슈뢰딩거 묘사의 시간에 따른 파동함수($\psi(t) = e^{-i\hat{H} t} \psi(\textbf{r})$)와 같은 형태를 가지게 됩니다.\\
이상의 내용을 수식으로 표현하면 다음과 같습니다.(아래부터 연산자에 $\hat{}$ 기호를 생략함)
\subsection*{Interaction picture}
\begin{align*}
    \text{Hamiltonian : }  \quad & H = H_0 + H' \\
    \text{Wavefunction : }\quad &\psi_{i} = e^{iH_0t}\psi \\
    \text{Interaction Hamiltonian : } \quad & A(t)_{int} = e^{-iH_0t}A_{int}e^{iH_0t}
\end{align*}
 이 때의 시간에 따른 파동함수의 변화와 상호작용 해밀토니안은 다음과 같은 방정식으로 표현됩니다.
 \begin{align}
    i\frac{\partial{\psi_i}}{\partial{t}} &= A_{int}(t)\psi_i\\
      A_{int}(t)&=e^{iH_0 t}A_{int}e^{iH_0t}
 \end{align}
 각 방정식의 유도는 다음과 같은 과정을 따릅니다:
 \begin{align*}
    \text{Derivation of time-dependent Wavefunction : } \quad  i\frac{\partial{\psi_i}}{\partial{t}} 
    &= i\frac{\partial}{\partial{t}}e^{iH_0 t}\psi \\
    & = i(iH_0 e^{iH_0 t} \psi + e^{iH_0 t} \frac{\partial{\psi}}{\partial{t}}) \\
    & = i(iH_0 e^{iH_0 t} \psi - i e^{iH_0 t} H \psi) \\
    & = i(iH_0 e^{iH_0 t} \psi - i e^{iH_0 t} (H_0 + H_{int}) \psi) \\
    & = e^{iH_0t}H_{int}\psi\\
    & = e^{iH_0t}H_{int}e^{-iH_0t}\psi_i
 \end{align*}
 이 때 상호작용 묘사와 슈뢰딩거 묘사에서의 파동함수 관계식인 $\psi_i = e^{iH_0t}\psi$ 와, 슈뢰딩거 방정식 $ i\frac{\partial}{\partial{t}}\psi = H\psi $ 
 를 이용하였습니다. 따라서 (1)과 (2)의 식이 유도됩니다.
 \subsection*{S matrix derivation}
 이제 상호작용하는 해밀토니안에 의한 파동함수의 시간에 따른 변화를 구하기 위해, 미분방정식 $i\frac{\partial{\psi_i}}{\partial{t}} = e^{iH_0t}H_{int}e^{-iH_0t}\psi_i$ 의 해를 찾아보기로 합니다.
 미분방정식의 양변을 시간 t에 대해 적분하면 다음과 같은 적분방정식이 됩니다.
 \begin{align}
   \psi_i(t)=\psi_i(t_0)-i\int^t_{t_0}H_{int}(t')\psi_i(t')dt'
 \end{align}
 이 적분방정식의 해를 다음과 같이 근사식들을 이용한 급수해라고 가정합니다.
 \begin{align}
    \psi_i(t) = \psi^0_i(t) + \psi^1_i(t) + \dots + \psi^n_i(t)
 \end{align}
 0차 근사는 다음과 같은 가정에서 이루어집니다 : 어떤 시간에서의 $\psi_i(t)$ 값을 이미 잘 알고 있다고 한다면,
 \begin{align}
    \psi_i(t_0) = \psi^0_i(t)
 \end{align}
 과 같이 쓸 수 있고, 1차근사를 계산하기 위해 슈뢰딩거 방정식에 주어진 파동함수를 대입하면,
 \begin{align}
    i\frac{\partial}{\partial{t_1}}\psi_i(t_1) &= H_{int}(t_1)\psi_i(t_1)\\
    \int^{t}_{t_0} dt_1 i\frac{\partial}{\partial{t_1}}\psi_i('t) &= -i\int^{t}_{t_0} dt_1H_{int}(t_1)\psi^0_i(t_0) \\
    \psi_i(t)-\psi_i(t_0) &= -i\int^{t}_{t_0} dt_1H_{int}(t_1)\psi_i(t_1) \\
    \psi_i(t) 
    &=\psi_i(t_0) - \underbrace{i\int^{t}_{t_0} dt_1H_{int}(t_1)\psi^0_i(t_0)}_\text{1st approximation}
 \end{align}
 2차 근사를 계산하기 위해 슈뢰딩거 방정식에 적분방정식을 다시 대입하면,
 \begin{align}
   i\frac{\partial}{\partial{t_1}}\psi_i(t_2) &= H_{int}(t_1)\psi_i(t_1)\\
      &=H_{int}\bigg(\psi_i(t_0) -i \int^{t}_{t_0} dt_1H_{int}(t_1)\psi^0_i(t_0)\bigg)\\
      &=H_{int}\psi_i(t_0) -i H_{int}\int^{t}_{t_0} dt_1 H_{int}(t_1)\psi^0_i(t_0)\\
   \int^{t}_{t_0} dt_1 i\frac{\partial}{\partial{t_1}}\psi_i('t) 
      &= \int^t_0 dt_1 H_{int}\psi_i(t_0) -i \int^t_0H_{int}\int^{t_1}_{t_0} dt_2 H_{int}(t_1)\psi^0_i(t_0) \\
   \psi_i(t)-\psi_i(t_0) 
      &= -i\int^t_0 dt_1 H_{int}\psi_i(t_0) - \int^t_0H_{int}\int^{t_1}_{t_0} dt_2 H_{int}(t_1)\psi^0_i(t_0) \\
   \psi_i(t) &= \psi_i(t_0) -i \int^t_0 dt_1 H_{int}\psi_i(t_0) - \underbrace{\int^t_0H_{int}\int^{t_1}_{t_0} dt_2 H_{int}(t_1)\psi^0_i(t_0)}_\text{2nd approximation}
\end{align}
위와 같은 방법으로 n차 근사를 구하면 다음과 같이 구해집니다.
\begin{align}
   \psi^n_i(t) = (-i)^n \int^t_{t_0}H_{int}(t_1)dt_1\int^{t_1}_{t_0}H_{int}(t_2)dt_2\cdots \int^{t_{n-1}}_{t_0}H_{int}(t_n)dt_n\psi_i(t_0)
\end{align}
따라서 상호작용 해밀토니안에 의한 시간에 따른 파동함수의 변화는 S-matrix를 사용하여:
\begin{align}
   \psi_i(t) = S(t,t_0)\psi_i(t_0)
\end{align}
이 때의 S-matrix는 다음과 같습니다:
\begin{align}
   S(t,t_0)=1-\underbrace{i\int^t_{t_0}H_{int}(t_1)dt_1}_\text{1st term} + \cdots + \underbrace{(-i)^n \int^t_{t_0}H_{int}(t_1)dt_1\cdots\int^{t_{n-1}}_{t_0}H_{int}(t_n)dt_n}_\text{nth term}
\end{align}
이제 시간의 구간을 나열하는 경우의 수를 생각해봅니다. n번째 항에서 시간 구간이 들어갈 수 있는 곳을 $O_i, i=(1,2,...,n)$으로 표시하면 다음과 같습니다.
\begin{align}
   \underbrace{(i)^n \int_{O_1}^{O_1} H_{int}(t_{O_1})dt_{O_1} \cdots \int_{O_n}^{O_n} H_{int}(t_{O_n})dt_{O_n} }_\text{총 n개의 O}
\end{align}
즉 $O_i$안에 들어갈 수 있는 시간 구간을 나열하는 경우의 수는 n!개가 됩니다. 따라서 S matrix를 구성하는 각 항에 대해 시간 구간을 나열하는 경우의 수를 표시하면,
\begin{align}
   S(t,t_0) =& 1 - \underbrace{i\int_{O}^{O} H_{int}(t_O)dt_O}_\text{1개의 시간구간} -\underbrace{\frac{1}{2!}\int_{O}^{O} H_{int}(t_O)dt_O\int_{O}^{O} H_{int}(t_O)dt_O}_\text{2개의 시간구간} \\
   & + i\frac{1}{3!}\underbrace{\int_{O}^{O} H_{int}(t_O)dt_O\int_{O}^{O} H_{int}(t_O)dt_O\int_{O}^{O} H_{int}(t_O)dt_O}_\text{3개의 시간구간} \\
   & + \cdots \underbrace{\frac{1}{n!}(-i)^n\int_{O}^{O} H_{int}(t_O)dt_O\cdots\int_{O}^{O} H_{int}(t_O)dt_O}_\text{n개의 시간구간}
\end{align}
이 때 시간을 나열할 수 있는 경우의 수 중 왼쪽에서 오른쪽으로 갈 수록 더 큰 시간의 상한선을 나타내게끔 하기 위해 배열 연산자 T를 도입하면, 예를 들어, S matrix의 n차 항을 $S^{(n)}$이라 하면
\begin{align}
   S^{(n)} &=\frac{(-i)^n}{n!}\int^{t}_{t_0}\cdots\int^t_{t_0}T\{H_{int}(O)\cdots H_{int}(O)\} dt_O \cdots dt_O \\
      &= \frac{(-i)^n}{n!}\int^{t}_{t_0}\cdots\int\{H_{int}(t_1)\cdots H_{int}(t_n)\} dt_1 \cdots dt_n \quad \text{where} \quad (t_1<t_2<\cdots <t_n)
\end{align}
그리고 이 때 S matrix는 다음과 같이 쓸 수 있습니다:
\begin{align}
   S(t,t_0)=T e^{-i\int_{t_0}^{t}H_{int}(t')dt'}
\end{align}
\subsection*{Operators using S matrix}
첫번째로, S matrix는 다음과 같은 성질을 가집니다.
\begin{align}
   S(t_2,t_1)S(t_1,t_0)=S(t_2,t_0), \quad t_2>t_1>t_0
\end{align}
위 식을 풀어쓰면, 
\begin{align}
   S(t_2,t_1)S(t_1,t_0) &= e^{-i\int_{t_2}^{t_1}H_{int}(t')dt'}e^{-i\int_{t_0}^{t_1}H_{int}(t')dt'} \\
      & = e^{-i\int_{t_2}^{t_0}H_{int}(t')dt'}\\
      & = S(t_2,t_0)
\end{align}
이제 식 (17) $\psi_i(t) = S(t,t_0)\psi_i(t_0)$ 에서, 파동함수를 변화시키는 S matrix 연산자와 같은 성질의 다른 연산자 $Q(t)=S(t,t_0)Q(t_0)$를 가정해봅시다. 식 (17)은 따라서:
\begin{align}
   \psi_i&=Q(t)\psi
\end{align}
가 됩니다. 시간 t는 S matrix operator의 적분 상한선인 t값에만 영향을 받으므로, 임의의 시간에 무관한 연산자 P를 이용해 $Q(t)$를 고쳐쓰면 다음과 같습니다.
\begin{align}
   \psi_i=S(t,\alpha)P\psi
\end{align}
P의 값을 알아내기 위해 
\\1. $\psi=e^{i\hat{H}t}\psi_s$, $H=H_0 + H_{int}$에서 $\psi_i=e^{iH_{int}t}\psi_s$를 이용하면 :
\begin{align}
   e^{iH_0t}=S(t,\alpha)P e^{iHt}
\end{align}
2. $S(\alpha,\alpha)=1$ 을 이용하면:
\begin{align}
   P=e^{iH_0\alpha}e^{-iH\alpha}=e^{-iH_{int}\alpha}
\end{align}
이 됩니다. \\
이제 중요한 가정인, 어떠한 이벤트(상호작용)이 일어나기 오래 전의 시간을 $-\infty$로 둡니다. 이 시점에서 입자간의 상호작용이 일어나지 않았다고 가정할 경우, 시간 t에서 입자의 상호작용은 다음과 같이 표현할 수 있습니다.
\begin{align}
   \psi_i&=S(t,\alpha)P\psi,\quad P\rightarrow 1\\
      &\leftrightarrow S(t)\psi,\quad\text{where}\quad S(t)=S(t,-\infty)
\end{align}
이 결과로부터 S matrix Operator의 성질 두번째를 이끌어낼 수 있습니다.
\begin{align}
   &S(t_2,t_1)S(t_1,-\infty)=S(t_2,t_1)S(t_1)=S(t_2,-\infty) = S(t_2)\\
   &\leftrightarrow S(t_2,t_1)=S(t_2)S^{-1}(t_1)
\end{align}
\subsection*{Time-ordering and Groundstate Average}
하이젠베르크 묘사에서의 연산자를 상호작용 묘사에서의 S matrix Operator를 이용해 표현하면 다음과 같습니다.
\begin{align}
   \hat{A}_h(t)=S^{-1}(t)A(t)S(t)
\end{align}
어떤 계가 있을 때, 계의 바닥상태를 나타내는 파동함수를 $\ket{\nu}$라고 합시다. 시간에 따른 계의 변화를 계의 바닥상태를 기준으로 기댓값을 구하면 다음과 같습니다.
\begin{align}
   \bra{\nu}T[\hat{A}(t)\hat{B}(t')\hat{C}(t'')\cdots]\ket{\nu}
\end{align}
이 때의 T는 Time-ordering operator로, $t>t'>t''>\cdots$의 순서로, 높은 시간 상한선일 수록 왼쪽으로 가게 나열해줍니다.
\end{document}