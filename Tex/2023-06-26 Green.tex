\documentclass{article}
\usepackage[fleqn]{amsmath}
\usepackage{physics}
\usepackage{mathrsfs}
\usepackage{amssymb}
\usepackage{microtype}
\newcommand{\RN}[1]{%s
  \textup{\uppercase\expandafter{\romannumeral#1}}%
}
\usepackage{geometry}
\geometry{
 a4paper,
 left=20mm,
 right=20mm,
 top=15mm,
 bottom=20mm
 }
\begin{document}
\title{Ch.9 Equation of motion (1)}
\author{Ha Nul Seok}
\maketitle
\section{Bruus}
\subsection{Derivation of Green's function}
The given schrodinger equation is:
\begin{equation*}
    [H_0(\textbf{r})+H_{i}(\textbf{r})]\Psi_E=\Psi_E
\end{equation*}
Define the corresponding Green's function as :
\begin{equation*}
    [E-H_0(r)]G_0(\textbf{r},\textbf{r}'\textit{E})=\delta(\textbf{r}-\textbf{r}')
\end{equation*}
Once more, Define the inverse $G_0^{-1}(\textbf{r},\textit{E})=E-H_0(\textbf{r})$ , Then
\begin{equation*}
    G_0^{-1}(\textbf{r},\textit{E})G_0(\textbf{r},\textbf{r}',\textit{E})=\delta(\textbf{r}-\textbf{r}')
\end{equation*}
Then Schrodinger equation rewritten as:
\begin{flalign*}
[G_0^{-1}(\textbf{r},\textit{E})-H_{i}(\textbf(r))]\Psi_E&=0\\
    &=(G_0^{-1}(\textbf{r},\textit{E})-H_i(\textbf{r}))\Psi_E=0\\
    &=G_0^{-1}(\textbf{r},\textit{E})\Psi_E-H_i(\textbf{r})\Psi_E=0\\
    &=\delta(\textbf{r}-\textbf{r}')G_0^{-1}(\textbf{r},\textbf{r}'E)\Psi_E-H_i(\textbf{r})\Psi_E=0\\
    &=\delta(\textbf{r}-\textbf{r}')\Psi_E=G_0(\textbf{r},\textbf{r}'E)H_i(\textbf{r})\Psi_E
\end{flalign*}
Then solution may be written as an integral equation
\begin{equation*}
    \Psi_
    E(\textbf{r})=\Psi_E^0(\textbf{r})+\int\textbf{r}'G_0(\textbf{r},\textbf{r}',\textit{E})H_i(\textbf{r}')\Psi_E(\textbf{r}')
\end{equation*}
Solving integral equation by iteration, and up to first order:
\begin{equation*}
    \Psi_E(\textbf{r})=\Psi_E^0(\textbf{r})+\int\textbf{r}'G_0(\textbf{r},\textbf{r}',\textit{E})H_i(\textbf{r}')\Psi^0_E(\textbf{r}') + \mathcal{O}(\textit{V}^2)
\end{equation*}
Now based upon above discussion, in the case of time-dependent Hamiltonian, $H=H_0+H_i$,
\begin{equation*}
    [i\partial_t-H_0(\textbf{r})-H_{i}(\textbf{r})]\Psi(\textbf{r},t)=0
\end{equation*}
Define Green's function by:
\begin{flalign*}
    &[i\partial_t-H_0(\textbf{r})]G_0(\textbf{r}t,\textbf{r}'t')=\delta(\textbf{r}-\textbf{r}')\delta(t-t')\\
    &[i\partial_t-H_0(\textbf{r})-H_i(\textbf{r})]G(\textbf{r}t,\textbf{r}'t')=\delta(\textbf{r}-\textbf{r}')\delta(t-t')
\end{flalign*}
Then inverse of Green's function is:
\begin{flalign*}
    &G_0^{-1}(\textbf{r},t)=i\partial_t-H_0(\textbf{r})\\
    &G^{-1}((\textbf{r},t))=i\partial_t-H_0(\textbf{r})-H_i(\textbf{r})
\end{flalign*}
Adjust as a integral equation form:
\begin{flalign*}
    &G_0^{-1}(\textbf{r},t)G_0(\textbf{r}t,\textbf{r}'t')=\delta(\textbf{r}-\textbf{r}')\delta(t-t')\\
    &G^{-1}((\textbf{r},t))G(\textbf{r}t,\textbf{r}'t')=\delta(\textbf{r}-\textbf{r}')\delta(t-t')\\
    \\
    &G_0^{-1}(\textbf{r},t)\Psi(\textbf{r},t)-H_{i}(\textbf{r})\Psi(\textbf{r},t)=0\\
    &G^{-1}(\textbf{r},t)\Psi(\textbf{r},t)-H_{i}(\textbf{r})\Psi(\textbf{r},t)=0
    \\
    \\
    &\delta(\textbf{r}-\textbf{r}')\delta(t-t')\Psi(\textbf{r},t)=G_0(\textbf{r}t,\textbf{r}'t')\Psi(\textbf{r},t)\\
    &\delta(\textbf{r}-\textbf{r}')\delta(t-t')\Psi(\textbf{r},t)=G(\textbf{r}t,\textbf{r}'t')H_{i}(\textbf{r})\Psi(\textbf{r},t)
\end{flalign*}
Solution to time-dependent Schrodinger equation,
\begin{equation*}
    \Psi(\textbf{r},t)=\Psi^0(\textbf{r},t)+\int d\textbf{r}'\int dt' G_0(\textbf{r},\textbf{r}';t,t')H_i(\textbf{r}')\Psi^0(\textbf{r}',t')
\end{equation*}
\subsection{Fourier transforms of advanced and retarded functions}
Retarded : a physical observable due to the action of some force or interaction at times prior to the measurement\\
This means that the observing(standard of calculation) time $t'$ must be small than the measurment time $t$,
General form of retarded function is : 
\begin{flalign*}
    C^R_{AB}(t,t')&=-i\theta(t-t') \bigg< \big[ A(t),B(t')\big] \bigg>
                \\ &=-i\theta(t-t') \bigg< A(t)B(t')-B(t')A(t) \bigg>
\end{flalign*}
Thermal average of $A(t)B(t')$ is, because of the time-dependent operator of interaction picture can be written as:
\begin{flalign*}
    \hat{O}(t)=e^{iH_0t}\hat{O}e^{-iH_0t'}
\end{flalign*}
Thus,
\begin{flalign*}
    \langle A(t)B(t') \rangle &= \frac{1}{Z} Tr \bigg[e^{\beta H_0}e^{iH_0t}Ae^{-iH_0t'}e^{iH_0t}Be^{-iH_0t'}\bigg]
    \\ &=\frac{1}{Z} Tr \bigg[e^{-iH_0t'}e^{\beta H_0}e^{iH_0t}Ae^{-iH_0t'}e^{iH_0t}B\bigg]
    \\ &=\frac{1}{Z} Tr \bigg[e^{\beta H_0}e^{iH_0(t-t')}Ae^{iH_0(t-t')}B\bigg]
\end{flalign*}
$C_{AB}$ depends on time variable, Fourier transform of $C^R(t-t')$ is :
\begin{flalign*}
    C^R_{AB}(\omega)=\int^\infty_\infty dt e^{i\omega t}C^R_{AB}(t)
\end{flalign*}
This is available when time difference $t-t' \rightarrow \infty$. Guess $t'=0$ and because of long time interval, there are no interaction between two time dependent operators,
\begin{flalign*}
    C^R_{AB}(t,t') &=i\theta(t-0)\bigg(\langle A(t)B(t') \rangle- \langle B(t')A(t) \rangle \bigg)\\
                &=\bigg(\langle A(t)B(t') \rangle- \langle B(t')A(t) \rangle \bigg)\\
                &=\bigg(\langle A(t)\rangle \langle B(t')\rangle - \langle B(t')\rangle \langle A(t) \rangle \bigg)\\
                &=C^R_{AB}(t)
\end{flalign*}
\section{Abriskov}
\subsection{Derivation of Green's fucntion}
From the change of basis, the Quantum field operator is written by using eigenfunction which notes 'some state $\nu$',
and 'position basis $\textbf{r}$',
\begin{flalign}
    \Psi(\textbf{r})\equiv\sum_\nu\braket{r}{\psi_\nu}a_\nu=\sum_\nu\psi_\nu(r)a_\nu \quad , \quad \Psi^\dagger(\textbf{r})\equiv\sum_\nu\braket{r}{\psi_\nu^*}a^\dagger_\nu=\sum_\nu\psi^*_\nu(r)a^\dagger_\nu
\end{flalign}
and its Fourier transform for momentum basis k is :
\begin{flalign}
    \Psi(\textbf{r}) \equiv\sum_k e^{i\textbf{k$\cdot$r}} a_\nu \quad , \quad \Psi^\dagger(\textbf{r})\equiv\sum_k e^{-i\textbf{k$\cdot$r}} a^\dagger_\nu
\end{flalign}
Now Using these features, lets begin the derivation for Green's function.
The one-particle operator $F^{(1)}$can be written as:
\begin{flalign}
    F^{(1)}=\int \Psi^\dagger(\xi)f^{(1)}\Psi(\xi) d\xi
\end{flalign}
Now we are treating the free-particle system, and guess that the Hamiltoniam become a quadratic form,
\begin{flalign}
    H=\sum\epsilon_0(p)n_p = \sum \epsilon_0 a^\dagger_p a_p
\end{flalign}
according to equation (2),and guess that the $\psi$ is the free-particles eigenfunction, $\Psi$ in the Heisenberg picture can be described as
\begin{flalign}
    \hat{\Psi}(r,t) &= \frac{1}{\sqrt{V}} \sum_p e^{i\sum_{p'}\epsilon_0(p')n_p t} a_p e^{-i\sum_{p"}\epsilon_0(p")n_p t} = \frac{1}{\sqrt{V}} \sum_p e^{iH t} a_p e^{-iH t}\\
        &=\frac{1}{\sqrt{V}} \sum_p e^{i[(p\cdot t)-\epsilon_0(p)t]}
\end{flalign}
Now, the Definition of Green's function is,
\begin{flalign}
    G_{\alpha \beta}(x,x')=-i\langle T(\hat{\Psi}_\alpha(x)\hat{\Psi}^\dagger_\beta(x'))\rangle
\end{flalign}
Using the Green's function, the average over the ground state of any one-particle operator is :
\begin{flalign}
    F^{(1)}=\pm i \int d^3\textbf{r} \bigg[\lim_{x'\rightarrow x+0}f^{(1)}_{\alpha\beta}(x)G_{\alpha \beta}(x,x')\bigg]
\end{flalign}
here $x=\{\textbf{r},r\}$.Same method, expand G into a Fourier integral :
\begin{flalign}
    G(x-x')=\int \frac{d^4p}{(2\pi)^4} G(p,\omega) e^{i[(p\cdot r -r')-\omega(t-t')]} (d^4p = d^3\textbf{p}d\omega)
\end{flalign}
\end{document}