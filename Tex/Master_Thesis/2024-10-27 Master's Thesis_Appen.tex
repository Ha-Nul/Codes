\documentclass{article}[12pt]
\usepackage{physics}
\usepackage{setspace}
\usepackage{amsmath}
\usepackage{mathrsfs}
\usepackage{amssymb}
\usepackage{feynmp-auto}
\usepackage{tgtermes}
\usepackage{graphicx}
\usepackage{booktabs}
\usepackage{array}
\usepackage{caption}
\usepackage{listings}
\usepackage{xcolor}
\usepackage{helvet}
\usepackage{float}
\definecolor{codegreen}{rgb}{0,0.6,0}
\definecolor{codegray}{rgb}{0.5,0.5,0.5}
\definecolor{codepurple}{rgb}{0.58,0,0.82}
\definecolor{backcolour}{rgb}{0.95,0.95,0.92}
\definecolor{lightgray}{rgb}{0.95,0.95,0.95}

\lstdefinestyle{mystyle}{
    backgroundcolor=\color{lightgray},   
    commentstyle=\color{codegreen},
    keywordstyle=\color{magenta},
    numberstyle=\tiny\color{codegray},
    stringstyle=\color{codepurple},
    basicstyle=\fontfamily{pcr}\selectfont\footnotesize,
    breakatwhitespace=false,         
    breaklines=true,                 
    captionpos=b,                    
    keepspaces=true,                 
    numbers=left,                    
    numbersep=5pt,                  
    showspaces=false,                
    showstringspaces=false,
    showtabs=false,                  
    tabsize=2
}
\setcounter{page}{46}
\setcounter{figure}{34}
\setcounter{table}{7}
\setcounter{section}{5}
\lstset{style=mystyle}
\numberwithin{equation}{section}
\captionsetup{font=footnotesize}
\newcommand{\RN}[1]{%s
  \textup{\uppercase\expandafter{\romannumeral#1}}%
}
\usepackage{geometry}
\geometry{
 a4paper,
 left=25.4mm,
 right=25.4mm,
 top=30mm,
 bottom=25.4mm
 }
\begin{document}

\section{Conclusion \& Future work}
\begin{spacing}{1.5}
\subsubsection*{Conclusion}
In conclusion, we’ve simulated the RSJJ Hamiltonian using the diagrammatic approximation method to determine its criticality. 
To consider the temperature dependency of the RSJJ system, we developed our idea in Matsubara formalism in steady-state conditions 
and calculated the partition function for consecutive temperatures. Chapter 2 deals with the representation methodology of field theory 
and how it expanded into a many-body problem with topological structure. A detailed description of the simulation model is in sec 2.1. We showed how the RSJJ Hamiltonian model 
is described in a macroscopic state of view. Operator $\hat{N}$ was used for conveying the bosonic interaction effect to the target system, 
and a detailed matrix form was derived in subsections. Sec 2.1 to Sec 2.2 introduced 
the basic method for solving many-body Hamiltonian. Specifically, 
we calculate the partition function for the thermal equilibrium state propagating in imaginary time. 
We describe the mapping procedure of our RSJJ model in the present pseudo-particle solver method in Sec 3.2.  
  In Chapter 4, the program code for implementing the simulation in a computational environment 
was analyzed. We evaluate the accuracy of our simulation approach and validate the approximate result at the beginning of Chapter 5, 
which is that calculated values converged toward the result from Exact diagonalization and higher order expansion method. 
The detailed procedure is explained in Sec 5.1. As a result of the simulation, 
criticality existed in the order parameter calculated in discrete temperature intervals, which can be deduced that the system feature goes on an insulating phase when temperature decreases, 
Whence in the previous study phase, a transition occurred in the fixed interaction parameter. 
We assumed this different phenomenon was revealed due to the finite temperature condition of our study. 
We also evaluated the approximation result, which was calculated from the correlation function, and found that it indicates different trends through temperature.
Yet it was hard to decide the exact critical point of our simulation progress, so more research should be done to get concrete results for examination. 
\subsubsection*{Future work}
First, as the most straightforward task, we can modify the range conditions for the $\alpha$ and $\gamma$ values and proceed with the calculations. By performing calculations for higher γ values, we can get more accurate data for the expected changes in criticality predicted by the order parameter.
The second task is to introduce higher-order approximation calculation methods. We have a calculation code capable of performing TOA, and using this code to perform calculations within the same parameter range is expected to yield more accurate results. However, it should be considered that this case requires significant time to proceed with the calculations.
Third, considering to extend the current framework to the Keldysh contour. The formalism employed in this study, the Matsubara formalism, transforms the time-dependent energy changes into temperature-dependent changes for the calculation. This formalism is vulnerable to considering the effects of non-equilibrium phenomena such as dissipation. We anticipate that more accurate calculations will be feasible under real-time analysis, incorporating non-equilibrium phenomena.
\end{spacing}
\pagebreak
\newpage
\appendix
\section{Circuit Hamiltonian}
\begin{spacing}{1.5}
From the circuit in Figure 3, we can establish the circuit equation for the Josephson junction.
\subsection{Josephson junction}
A Josephson junction consists of two conductors separated by a thin insulator(S-I-S structure), 
where the insulator acts as a potential barrier.
Two factors determine the energy of a Josephson junction: 
the energy derived from Cooper pairs tunneling through the potential barrier, analogous to the inductor 
energy in an LC circuit, and the effective capacitance due to the S-I-S structure of the junction.
The effective inductance is determined by the critical current $I_c$ and flux $\Phi = \int_0^t V(t) dt$, 
the potential difference $V$ across the junction, 
and the energy $E_J$ of the Cooper pairs that have crossed the potential barrier of the junction. 
This can be expressed as the following equation: 
\begin{flalign}
\begin{split}
\begin{cases} I = I_c \sin \Phi \\ \frac{\partial \Phi}{\partial t} = \frac{2e}{\hbar} V \end{cases}
\end{split}
\end{flalign}
Combining the two equations, we obtain the following relation for Cooper pairs:
\begin{flalign}
\begin{split}
\frac{\partial I}{\partial t} = \frac{\partial I}{\partial \Phi}\frac{\partial \Phi}{\partial t} = I_cV\frac{2e}{\hbar} \cos{\Phi} \\ \mathcal{L}_{J_J} = E_J\cos\frac{2e}{\hbar}\Phi
\end{split}
\end{flalign}
And the energy of the junction due to the effective capacitance is written as follows:
\begin{flalign}
\begin{split}
\mathcal{L}_{J_c} =\frac{C_J}{2}\dot{\Phi}^2
\end{split}
\end{flalign}
From this, the Lagrangian of the entire Josephson junction can be written as follows:
\begin{flalign}
\begin{split}
\mathcal{L}_J = \mathcal{L}_{J_c} + \mathcal{L}_{J_J} \\ = \frac{C_J}{2}\dot{\Phi}^2 + E_J\cos\frac{2e}{\hbar}\Phi
\end{split}
\end{flalign}
Performing Legendre transformation same as the case of LC oscillator, the Hamiltonian of the Josephson junction can be rewritten as follows:
\begin{flalign}
\begin{split}
H_{JJ} = \frac{{Q_J}^2}{2C_J} - E_J \cos\frac{2e}{\hbar}\Phi
\end{split}
\end{flalign}
This is equivalent to the Hamiltonian of a simple pendulum suspended in a gravitational field. 
$\cos\Phi$ term makes the Josephson junction behave like a nonlinear harmonic oscillator, 
where the Josephson energy $E_J$ controls the degree of nonlinearity.
\subsection{Full Hamiltonian}
By using the potential energy (voltage) applied to the circuit, the Lagrangian of the entire circuit can be expressed as follows.[18]
\begin{flalign}
\begin{split}
\mathcal{L} = \frac{C_J}{2}\dot{\Phi}^2 + E_J\cos{\frac{2e}{\hbar}} + \frac{C_C}{2}(\dot{\Phi}_1 - \dot{\Phi})^2 + \sum_{i=2}^N \frac{cx}{2} \dot{\Phi}_i^2 -\sum_{i=2}^N\frac{(\Phi_i - \Phi_{i-1})^2}{2lx}
\end{split}
\end{flalign}
To transform the Lagrangian into the Hamiltonian form, we define the variable  $Q_i = \frac{\partial \mathcal{L}}{\partial \dot{\Phi}_i}$ for the LC circuit, along with the previously defined canonical conjugate variables. This can be interpreted as the charge accumulated on the capacitor connected to the i-th node.
Considering the variable with respect to which the differentiation is performed, the total Lagrangian can be divided into the following two parts.
\begin{flalign}
  \begin{split}
\mathcal{L} = \mathcal{L}_{loc} + \mathcal{L}_{LC}
\end{split}
\end{flalign}
By applying appropriate Legendre transformations to each Lagrangian, the following form of the Hamiltonian is obtained.
\begin{flalign}
  \begin{split}
\hat{H} = \frac{(Q+Q_1)^2}{C_J} + \frac{Q^2_1}{2C_C}-E_J\cos(\frac{2e}{\hbar}\Phi) + \sum_{i=2}\frac{(\hat{\Phi}_i-\hat{\Phi}_{i-1})^2}{2lx} + \sum_{i=2}\frac{\hat{Q}_i^2}{2cx}
\end{split}
\end{flalign}
\subsubsection*{Transimission line composed by LC circuit}
The Lagrangian for a waveguide structure using an LC circuit consisting of inductors and capacitors corresponds to the following part.
\begin{flalign}
  \begin{split}
\mathcal{L}_{LC} = \sum_{i=2} \frac{cx}{2}\dot{\Phi}^2 - \sum_{i=2}\frac{(\Phi_i-\Phi_{i-1})^2}{2lx}
\end{split}
\end{flalign}
This originates from the form of energy stored in the capacitor due to the potential difference across it and the energy stored in the inductance per unit length of the inductance. By rewriting the Lagrangian into the Hamiltonian using the canonical variables for the LC circuit, the following equation can be obtained.
\begin{flalign}
  \begin{split}
H_{LC} = \sum_{i=2}\frac{(\hat{\Phi}_i-\hat{\Phi}_{i-1})^2}{2lx} + \sum_{i=2}\frac{\hat{Q}_i^2}{2cx}
\end{split}
\end{flalign}
\subsubsection*{Local system}
The Lagrangian for the capacitor connecti ng the Josephson junction and the LC oscillator can be written as follows.
\begin{flalign}
  \begin{split}
\mathcal{L}_{loc}=\frac{C_J}{2}\dot{\Phi}^2 + E_J\cos{\frac{2e}{\hbar}} + \frac{C_C}{2}(\dot{\Phi}_1 - \dot{\Phi})^2
\end{split}
\end{flalign}
The Legendre transformation for a multivariable function can be written as follows.
\begin{flalign}
  \begin{split}
Q &= \frac{\partial \mathcal{L}_{C_C}}{\partial \dot{\Phi}} = C_J \dot{\Phi}+ C_C(\dot{\Phi}-\dot{\Phi}_1) \quad , \quad \dot{\Phi} = \frac{Q}{C_J} - \frac{C_C}{C_J}(\dot{\Phi}-\dot{\Phi}_1) \\
Q_1 &= \frac{\partial \mathcal{L}_{C_C}}{\partial \dot{\Phi}_1} = \frac{C_C}{2}(2\dot{\Phi}_1-2\dot{\Phi}) \\
\end{split}
\end{flalign}
Using this, rewriting the Lagrangian into the Hamiltonian yields the following:
\begin{flalign}
  \begin{split}
H_{loc} = \frac{(Q+Q_1)^2}{C_J} + \frac{Q^2_1}{2C_C}-E_J\cos(\frac{2e}{\hbar}\Phi)
\end{split}
\end{flalign}
\subsubsection*{Quantization example - 1st node consideration}
For the case where it is the first node (i=1), when considering the variables, Hamiltonian can be written as follows.
\begin{flalign}
  \begin{split}
\hat{H}_1 &= \frac{(Q+Q_1)^2}{C_J} + \frac{Q^2_1}{2C_C}-E_J\cos(\frac{2e}{\hbar}\Phi) + \frac{\hat{\Phi}}{2lx} \\ 
&= \frac{1}{2}\begin{pmatrix}  \hat{P} & \hat{P}_1 \end{pmatrix}\begin{pmatrix} 1 & 0 \\ 0 & \sqrt{\frac{C}{C_p}}\end{pmatrix}^{-1}\begin{pmatrix} \frac{1}{C_J} & \frac{1}{C_J} \\ \frac{1}{C_J} & \frac{1}{C_P}\end{pmatrix}\begin{pmatrix} 1 & 0 \\ 0 & \sqrt{\frac{C}{C_p}}\end{pmatrix}\begin{pmatrix} \hat{P} \\ \hat{P}_1\end{pmatrix} + \\
&\frac{1}{2lx}\begin{pmatrix} \hat{\Psi} & \hat{\Psi}_1 \end{pmatrix}\begin{pmatrix} 1 & 0 \\ 0 & \sqrt{\frac{C}{C_p}}\end{pmatrix}\begin{pmatrix} 0 & 0 \\ 0 & 1 \end{pmatrix}\begin{pmatrix} 1 & 0 \\ 0 & \sqrt{\frac{C}{C_p}}\end{pmatrix}\begin{pmatrix} \hat{\Psi} \\ \hat{\Psi}_1 \end{pmatrix} -E_J\cos(\frac{2e}{\hbar}\Phi)
\end{split}
\end{flalign}
Here, C is a constant. If we expand and write the above Hamiltonian, it can be written as below.
\begin{flalign}
  \begin{split}
\hat{H}_1 = \frac{1}{2}\bigg(\frac{\hat{P}^2}{C_J} + \frac{2}{C_J}\sqrt{\frac{C_P}{C}}\big(\hat{P}\hat{P}_1\big) + \frac{\hat{P}_1}{C}\bigg) + \frac{1}{2lx}\frac{C}{C_P}\hat{\Psi}^2 - E_J\cos{\frac{2e}{\hbar}\hat{\Psi}}
\end{split}
\end{flalign}
The operators  $\hat{P}_1 , \hat{\Psi}_1$ can be rewritten as bosonic opeators using the following relation.
\begin{flalign}
  \begin{split}
\hat{P}_1 = -\sqrt{\frac{\hbar C \omega}{2}} (\hat{b}_1^\dagger + \hat{b}_1) \\ \hat{\Psi}_1 = i\sqrt{\frac{\hbar}{2C\omega}}(\hat{b}^\dagger - \hat{b})
\end{split}
\end{flalign}
Here, $\omega = \frac{1}{lxC_p}$. Using the above equation to rewrite the Hamiltonian, it can be rewritten into the following form:
\begin{flalign}
  \begin{split}
\hat{H}_1 \approx \frac{\hat{P}}{2C_J} - E_J\cos(\frac{2e}{\hbar}\hat{\Psi}) - g\hat{P}(\hat{b}^\dagger_1 +\hat{b}_1) + \hbar\omega\hat{b}^\dagger_1\hat{b}_1
\end{split}
\end{flalign}
Here, $g = \frac{1}{C_J}\sqrt{\frac{\hbar}{2lx}}$. This can be inferred the capacitor in the structure of the LC oscillator is not considered in the example circuit.
\end{spacing}
\pagebreak
\end{document}