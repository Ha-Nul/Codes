\documentclass{article}[12pt]
\usepackage{physics}
\usepackage{setspace}
\usepackage{amsmath}
\usepackage{mathrsfs}
\usepackage{amssymb}
\usepackage{feynmp-auto}
\usepackage{tgtermes}
\usepackage{graphicx}
\usepackage{booktabs}
\usepackage{array}
\usepackage{caption}
\usepackage{listings}
\usepackage{xcolor}
\usepackage{helvet}
\definecolor{codegreen}{rgb}{0,0.6,0}
\definecolor{codegray}{rgb}{0.5,0.5,0.5}
\definecolor{codepurple}{rgb}{0.58,0,0.82}
\definecolor{backcolour}{rgb}{0.95,0.95,0.92}
\definecolor{lightgray}{rgb}{0.95,0.95,0.95}

\lstdefinestyle{mystyle}{
    backgroundcolor=\color{lightgray},   
    commentstyle=\color{codegreen},
    keywordstyle=\color{magenta},
    numberstyle=\tiny\color{codegray},
    stringstyle=\color{codepurple},
    basicstyle=\fontfamily{pcr}\selectfont\footnotesize,
    breakatwhitespace=false,         
    breaklines=true,                 
    captionpos=b,                    
    keepspaces=true,                 
    numbers=left,                    
    numbersep=5pt,                  
    showspaces=false,                
    showstringspaces=false,
    showtabs=false,                  
    tabsize=2
}
\setcounter{page}{31}
\setcounter{figure}{22}
\setcounter{table}{3}
\lstset{style=mystyle}
\captionsetup{font=footnotesize}
\newcommand{\RN}[1]{%s
  \textup{\uppercase\expandafter{\romannumeral#1}}%
}
\usepackage{geometry}
\geometry{
 a4paper,
 left=25.4mm,
 right=25.4mm,
 top=30mm,
 bottom=25.4mm
 }
\begin{document}

\section*{6. Conclusion \& Future work}
\begin{spacing}{1.5}
    \subsubsection*{Conclusion}
In conclusion, we’ve simulated the RSJJ Hamiltonian using the diagrammatic approximation method to determine its criticality. 
To consider the temperature dependency of the RSJJ system, we developed our idea in Matsubara formalism in steady-state conditions 
and calculated the partition function for consecutive temperatures. Chapter 2, we dealt with the representation methodology of field theory 
and how it expanded into a many-body problem, with topological structure. A detailed description of the simulation model is in sec 2.1. We showed how the RSJJ Hamiltonian model 
is described in a macroscopic state of view. Operator $\hat{N}$ was used for conveying the bosonic interaction effect to the target system, 
detailed matrix form was derived in subsections.Sec 2.2 to Sec 2.3, 
the basic method for solving many-body Hamiltonian was introduced.More specifically, 
we calculate the partition function for the thermal equilibrium state that propagated in imaginary time. 
We describe the mapping procedure of our RSJJ model in the present pseudo-particle solver method in Sec 2.3.  
 In Chapter 3, the program code for implementing the simulation in a computational environment 
was analyzed. We evaluate the accuracy of our simulation approach and validate the approximate result at the beginning of Chapter 4, 
that calculated values converged toward the result from Exact diagonalization and higher order expansion method. 
Detailed procedure was explained in Sec 4.2 and Sec 4.3. As a result of the simulation, 
there were criticality existed in the order parameter calculated in discrete temperature intervals 
which can be deduced that the system feature goes on an insulating phase when temperature decreases, 
whence in the previous study phase transition occurs in fixed interaction parameter. 
We assumed this different phenomenon was revealed due to the finite temperature condition of our study. 
We also evaluated DC susceptibility and found that it indicates different trends through temperature. 
Yet deciding the critical point of phase transition is hard in our simulation conditions. 
\subsubsection*{Future work}
In the simulation aspect, we could set another range of simulation range for each temperature, 
$\alpha$, $\gamma$ condition to exceed the present parameter range and expect unobserved results 
due to the broad range of physical conditions. After we determine the concrete regime of system features, 
we could observe our system in a Non-equilibrium state following real-time contour to expand the pseudo-particle paradigm 
into many-body particle regions with more complex Josephson quantum circuit structures connected in geometrical order.
\end{spacing}
\pagebreak
\newpage
\section*{APPENDIX A. Perturbative field theory}
\begin{spacing}{1.5}
\subsection*{basic structure}

We can start our discussion in very basic perturbation field method which is adapting Wick’s theorem is available. To begin, we can define the partition function of configuration space that governed by the certain energy condition. The dynamic of variables of given space are governed under the action functional, thus partition function of space dictates the probability distribution of possible paths make system propagates onto. The partition function $Z[J]$ can be calculated in path integral method. For field $\phi(x,t)$ , corresponding Lagrangian is : 

\begin{flalign*}
\mathcal{L} =\frac{1}{2}\phi^TA\phi +\phi\cdot J \\ =\frac{1}{2}(-\phi^T \partial_t^2\phi + \phi^T \nabla^2\phi - m^2\phi) + J\phi
\end{flalign*}

Here, $A$ is a form of diagonal matrix. Consider only time in 4-vector notation, and use the above notation, we can write partition function in following:

\begin{flalign*}
Z[J]=\int[D\phi]e^{\int d\phi [\frac{1}{2}\partial^2_\tau\phi - m^2\phi^2]+J\cdot\phi}
\end{flalign*}

Which is, in short way:

\begin{flalign*}
Z[J]=Z[0]W[J]
\end{flalign*}

where $W[J]=e^{\frac{1}{2}J_iA^{-1}_{ij}J_j}$ , $Z[0] = \big(\frac{(2\pi i)^N}{\text{det}A}\big)^{\frac{1}{2}}$ . The sub indice $i,j$ means $i$th and $j$th element of $J(\phi)$. Based upon the probability distribution points of view, we can calculate the $n$th moment of the path distribution. In general, we can recognize that :

\begin{flalign*}
\langle \phi^{2n} \rangle = \frac{\int [D\phi] \phi^{2n} e^{\frac{1}{2} \phi^i A_{ij} \phi^j + J\cdot \phi}}{Z[0]} = \frac{\Pi_{i}\frac{ \mathcal{\delta}}{\mathcal{\delta}J_i}\frac{ \mathcal{\delta}}{\mathcal{\delta}J_j}\int[D\phi]e^{-\frac{1}{2}(J_{ij}A_{ij}J_{j})}}{Z[0]}= \Pi_{i}\frac{ \mathcal{\delta}}{\mathcal{\delta}J_i}\frac{ \mathcal{\delta}}{\mathcal{\delta}J_j} \ln{Z[J]}
\end{flalign*}

In general, this form of equation is constructed when exterior force or reaction caused to free propagating field. Now if there are some interaction happened between the field and the environment, especially in Lorentz-invariant action, we can treat its effect as a weak perturbation. Let $-\frac{g}{4}\phi^4$ is the perturbation term. Then the partition function turns out to be:

\begin{flalign*}
Z[J]=\int[D\phi]e^{\int d\phi [\frac{1}{2}\partial^2_\tau\phi - m^2\phi^2]+J\cdot\phi -\frac{g}{4}\phi^4}
\end{flalign*}

In same formula, we can write :

\subsection*{Perturbation theory and approximation method}

Guess if there are infinitesimal “real-time” variance appears in external Hamiltonian $H'$. We will treat it as a perturbation effect in the given Hamiltonian. In real-time procedure, let 

\begin{flalign*}
H'(t) = H'(t)+ \mathcal{\delta}H'(t)
\end{flalign*}

Using the above concept, We can rewrite the equation of motion for total Green’s function into : 

\begin{flalign*}
\{i \frac{\partial}{\partial t_1} +  \frac{\nabla^2_1}{2m} - H'(t_1) \pm \int dr_2 v(r_1-r_2)[G(t_1, t_1^+ ; H') + \frac{\mathcal{\delta}}{\mathcal{\delta} H'(t_1^+)}]\}G(t_1,t_1';H') = \mathcal{\delta} (t_1- t_1')
\end{flalign*}

Summarizing the result using local green’s function, we can get : 

\begin{flalign*}
G(t,t';H') =& G_0(t,t';H') \pm i \int^{-i\beta}_0 d\tau_1 d\tau_2 G_0(t,\tau_1;H')V(\tau_1-\tau_2) \\ 
&\bigg[G_0(\tau_2,\tau_2^+;H') + \frac{\mathcal{\delta}}{\mathcal{\delta} H'(\tau_2)}\bigg]G(\tau_1,t';H')
\end{flalign*}

This term can recursively solvable using the inverse matrix form of the bare Green’s function $G_0$,

\begin{flalign*}
G^{-1}(t,t';H') = \bigg[i\frac{\partial}{\partial t_1} + \frac{\nabla^2_1}{2m} - H(t)\bigg]\delta(t-t')
\end{flalign*}

And, 

\begin{flalign*}
\mathcal{\delta}G_0 = -G_0\mathcal{\delta}G_0^{-1}G_0
\end{flalign*}

Skipping the specific procedure, the final result of total Hamiltonian can be expanded to bare Green’s function with each time intervals : 

\begin{flalign*}
G(t,t';H') = G_0(t,t';H') \pm i\int^{-i\beta}_0 d\tau_1 d\tau_2 G_0(t,\tau_1;H')V(\tau_1-\tau_2)\bigg[G_0(\tau_2,\tau_2^+;H')G_0(\tau_1,t';H') \pm G_0(\tau_1,\tau_2^+;H')G_0(\tau_2;t_1;H')\bigg] 
\end{flalign*}

From the above derivation, we can get very basic two diagrams as and example of the result from the expansion. Corresponding diagrams are :

We can get more expanded term substitute the result recursively into $(***)$.

If we consider the interaction term with exterior conditions, we can use the concept of self-energy $\Sigma$ to describe the system’s dynamics. The self-energy $\Sigma$ is defined by : 

\begin{flalign*}
(i\frac{\partial}{\partial t_1} + H_0)G(t_1-t_1') - \int^{-i\beta}_0 d\tau_1 \Sigma(t_1-\tau_1)G(\tau_1-t_1) = \delta(t_1-t_1')
\end{flalign*}

And 

\begin{flalign*}
\Sigma(t,t';H')=\pm \int d\tau_2 V(t-\tau_2)G(\tau_2,\tau_2^+,H')\delta(t-t') + i \int d\tau_2 d\tau_1 V(t-\tau_2)\bigg[\frac{\delta G(t,\tau_1;H')}{\delta H'(\tau_2)}\bigg] G^{-1}(\tau,t',H')
\end{flalign*}

Which can be compare with the equation $(*)$ , form of the equation brings out full Green’s function G,
\end{spacing}
\end{document}