\documentclass{article}[12pt]
\usepackage{physics}
\usepackage{setspace}
\usepackage{amsmath}
\usepackage{mathrsfs}
\usepackage{amssymb}
\usepackage{feynmp-auto}
\usepackage{tgtermes}
\usepackage{graphicx}
\usepackage{booktabs}
\usepackage{array}
\usepackage{caption}
\usepackage{listings}
\usepackage{xcolor}
\usepackage{helvet}
\usepackage{float}
\definecolor{codegreen}{rgb}{0,0.6,0}
\definecolor{codegray}{rgb}{0.5,0.5,0.5}
\definecolor{codepurple}{rgb}{0.58,0,0.82}
\definecolor{backcolour}{rgb}{0.95,0.95,0.92}
\definecolor{lightgray}{rgb}{0.95,0.95,0.95}

\lstdefinestyle{mystyle}{
    backgroundcolor=\color{lightgray},   
    commentstyle=\color{codegreen},
    keywordstyle=\color{magenta},
    numberstyle=\tiny\color{codegray},
    stringstyle=\color{codepurple},
    basicstyle=\fontfamily{pcr}\selectfont\footnotesize,
    breakatwhitespace=false,         
    breaklines=true,                 
    captionpos=b,                    
    keepspaces=true,                 
    numbers=left,                    
    numbersep=5pt,                  
    showspaces=false,                
    showstringspaces=false,
    showtabs=false,                  
    tabsize=2
}
\setcounter{page}{31}
\setcounter{figure}{22}
\setcounter{table}{3}
\lstset{style=mystyle}
\numberwithin{equation}{section}
\captionsetup{font=footnotesize}
\newcommand{\RN}[1]{%s
  \textup{\uppercase\expandafter{\romannumeral#1}}%
}
\usepackage{geometry}
\geometry{
 a4paper,
 left=25.4mm,
 right=25.4mm,
 top=30mm,
 bottom=25.4mm
 }
\begin{document}

\section*{6. Conclusion \& Future work}
\begin{spacing}{1.5}
    \subsubsection*{Conclusion}
In conclusion, we’ve simulated the RSJJ Hamiltonian using the diagrammatic approximation method to determine its criticality. 
To consider the temperature dependency of the RSJJ system, we developed our idea in Matsubara formalism in steady-state conditions 
and calculated the partition function for consecutive temperatures. Chapter 2, we dealt with the representation methodology of field theory 
and how it expanded into a many-body problem, with topological structure. A detailed description of the simulation model is in sec 2.1. We showed how the RSJJ Hamiltonian model 
is described in a macroscopic state of view. Operator $\hat{N}$ was used for conveying the bosonic interaction effect to the target system, 
detailed matrix form was derived in subsections.Sec 2.2 to Sec 2.3, 
the basic method for solving many-body Hamiltonian was introduced. More specifically, 
we calculate the partition function for the thermal equilibrium state that propagated in imaginary time. 
We describe the mapping procedure of our RSJJ model in the present pseudo-particle solver method in Sec 2.3.  
 In Chapter 3, the program code for implementing the simulation in a computational environment 
was analyzed. We evaluate the accuracy of our simulation approach and validate the approximate result at the beginning of Chapter 4, 
that calculated values converged toward the result from Exact diagonalization and higher order expansion method. 
Detailed procedure was explained in Sec 4.2 and Sec 4.3. As a result of the simulation, 
there were criticality existed in the order parameter calculated in discrete temperature intervals 
which can be deduced that the system feature goes on an insulating phase when temperature decreases, 
whence in the previous study phase transition occurs in fixed interaction parameter. 
We assumed this different phenomenon was revealed due to the finite temperature condition of our study. 
We also evaluated DC susceptibility and found that it indicates different trends through temperature. 
Yet it was hard to decide the exact critical point of our simulation progress, there should be more research to get concrete results for examination. 
\subsubsection*{Future work}
In the simulation aspect, we could set another range of simulation range for each temperature, 
$\alpha$, $\gamma$ condition to exceed the present parameter range and expect unobserved results 
due to the broad range of physical conditions. After we determine the concrete regime of system features, 
we could observe our system in a Non-equilibrium state following real-time contour to expand the pseudo-particle paradigm 
into many-body particle regions with more complex Josephson quantum circuit structures connected in geometrical order.
\end{spacing}
\pagebreak
\newpage
\appendix
\section{Brief statement for field theory}
\begin{spacing}{1.5}
In the main text, we calculated the partition function using the action $S$. Here, we provide a brief explanation of this approach.
To begin, we can define the partition function of Minkowski space that restriced by the certain energy condition. 
The dynamic of variables of given space are governed under the action functional, 
thus partition function of space dictates the probability distribution of possible paths make system propagates onto. 
The partition function $Z[J]$ can be calculated in path integral method. For field $\phi(x,t)$ , 
corresponding Lagrangian is : 
\begin{flalign}
  \begin{split}
\mathcal{L} &=\frac{1}{2}\phi^TA\phi +\phi\cdot J \\ &=\frac{1}{2}(-\phi^T \partial_t^2\phi + \phi^T \nabla^2\phi - m^2\phi) + J\phi
  \end{split}
\end{flalign}
Here, $A$ is a form of diagonal matrix and $J$ corresponds to the potential of the target system we want to describe. Consider only time in 4-vector notation, and use the above notation, we can write partition function in following:
\begin{flalign}
Z[J]=\int[D\phi]e^{\int d\phi [\frac{1}{2}\partial^2_\tau\phi - m^2\phi^2]+J\cdot\phi}
\end{flalign}
Which is, in short way:
\begin{flalign}
Z[J]=Z[0]W[J]
\end{flalign}
A similar form to the above can be found in equations (2.40), (2.41), and (2.58) which are presented in the main text. 
 Here $W[J]=e^{\frac{1}{2}J_iA^{-1}_{ij}J_j}$ , $Z[0] = \big(\frac{(2\pi i)^N}{\text{det}A}\big)^{\frac{1}{2}}$ . 
The sub indice $i,j$ means $i$th and $j$th element of $J(\phi)$.Here, note the indices refer to Minkowski space, thus encompassing not only spatial coordinates but also temporal indices.
we can recognize that :
\begin{flalign}
\langle \phi^{2n} \rangle = \frac{\int [D\phi] \phi^{2n} e^{\frac{1}{2} \phi^i A_{ij} \phi^j + J\cdot \phi}}{Z[0]} = \frac{\Pi_{i}\frac{ \mathcal{\delta}}{\mathcal{\delta}J_i}\frac{ \mathcal{\delta}}{\mathcal{\delta}J_j}\int[D\phi]e^{-\frac{1}{2}(J_{ij}A_{ij}J_{j})}}{Z[0]}= \Pi_{i}\frac{ \mathcal{\delta}}{\mathcal{\delta}J_i}\frac{ \mathcal{\delta}}{\mathcal{\delta}J_j} \ln{Z[J]}
\end{flalign}
This implies that if we perform the above calculation for a small time variation, we can re-express the equation as an expansion in time. Upon rotating time by $i$ in the complex plane, this result matches the form of the thermal partition function calculated in the main text.
Similar, if there are some interaction happened between the field and the environment we can treat its effect as a weak perturbation. Let $-\frac{g}{4}\phi^4$ represents the interaction effect. Then the form of partition function turns out to be:
\begin{flalign}
Z[J]=\int[D\phi]e^{\int d\phi [\frac{1}{2}\partial^2_\tau\phi - m^2\phi^2]+J\cdot\phi -\frac{g}{4}\phi^4}
\end{flalign}
Which can be calculated as the same procedure above, in a more complex structure.
\end{spacing}
\pagebreak
\end{document}