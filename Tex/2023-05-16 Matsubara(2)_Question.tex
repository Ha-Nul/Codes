\documentclass{article}
\usepackage[fleqn]{amsmath}
\usepackage{physics}
\usepackage{mathrsfs}
\usepackage{amssymb}
\usepackage{microtype}
\newcommand{\RN}[1]{%s
  \textup{\uppercase\expandafter{\romannumeral#1}}%
}
\usepackage{geometry}
\geometry{
 a4paper,
 left=20mm,
 right=20mm,
 top=15mm,
 bottom=20mm
 }
\begin{document}
\title{Matsubara Summary(2) : About $\tau$}
\maketitle
To answering the question : What is the reason that $\tau$ has its boundary condition as : '$0<\tau<\beta$'?
Following the derivation in the textbook, written by Bruus and K, Ch.11 section 2, 
\\First, retarded single particle Green's function can be written as: 
\begin{flalign*}
    C^R_{AB}(\omega) = \frac{1}{Z} \sum_{nn'} \frac{\bra{n} A \ket{n'} \bra{n'} B \ket{n}}{\omega+E_n -E_{n'} + i\eta} \bigg( e^{\beta E_n}-(\pm)e^{-\beta E_{n'}} \bigg)
\end{flalign*}
And by the definition of Matsubara frequency,
\begin{flalign*}
    C_{AB}(\tau) = -\frac{1}{Z} & Tr[e^{\beta H}e^{\tau H} A e^{-\tau H}B] \\
                                &= -\frac{1}{Z} \int \psi^{0*}_n(x) e^{\beta H} e^{\tau H} a^\dagger e^{-\tau H} a \psi^{0}_n(x) dx \\
                                &= -\frac{1}{Z} \sum_n e^{\beta E_n}\int \psi^{0*}_n(x) e^{\tau H} a^\dagger e^{-\tau H} a \psi^{0}_n(x) dx \\
                                &= -\frac{1}{Z} \sum_n e^{\beta E_n}\int \psi^{0*}_n(x) a^\dagger \ket{\psi_{n'}}\bra{\psi_{n}} a \psi^{0}_n(x) dx e^{-\tau E_n - E_{n'}} 
\end{flalign*}
\begin{flalign*}
    C_{AB}(\tau) &= -\frac{1}{Z} Tr[e^{\beta H} e^{\tau H} A e^{-\tau H} B] \\
    &= -\frac{1}{Z} \sum_{nn'} e^{\beta H} \bra{n} A \ket{n'} \bra{n'} B \ket{n} \bigg( e^{\beta E_n}-(\pm)e^{-\beta E_{n'}} \bigg) \\
\end{flalign*}
Using Fourier Transformation,
\begin{flalign*}
    C_{AB} &= \int^\beta_0 d\tau e^{i\omega_n \tau}[-\frac{1}{Z} \sum_{nn'} e^{\beta H} e\bra{n} A \ket{n'} \bra{n'} B \ket{n} \bigg( e^{\beta E_n}-(\pm)e^{-\beta E_{n'}} \bigg)] \\
    &= \frac{1}{(E_n - E_-{n'} + \tau \omega_n i)}(-\frac{e^{\beta H}}{Z} c) e^{\tau(E_n - E_{n'} + i\omega+n)}|^\beta_0
\end{flalign*} 
Therefore,
\begin{flalign*}
    C_{AB} = \frac{1}{Z}\sum_{nn'}\frac{\sum_{nn'}  \bra{n} A \ket{n'} \bra{n'} B \ket{n}}{i \omega_n +E_n - E{n'}} (e^{beta E_n} - \pm e^{\beta E_{n'}})
\end{flalign*}
Consider the function in entire complex plane, where $z=x+iy$,
\begin{flalign*}
    C_{AB} = \frac{1}{Z}\sum_{nn'}\frac{\sum_{nn'}  \bra{n} A \ket{n'} \bra{n'} B \ket{n}}{z +E_n - E{n'}} (e^{beta E_n} - \pm e^{\beta E_{n'}})
\end{flalign*}
According to the theory of analytic functions : if two functions coincide in an a infinite set of points then they are fully identical functions within the entire domain where at least one of them is analyitc function.\\
That is, if $C_{AB}(i\omega)$ is known, then $C^R_{AB}(\omega)$ can be found :
\begin{flalign*}
    C^R_{AB}(\omega) = C_{AB}(i\omega \rightarrow \omega + i\eta)
\end{flalign*}
Where $\eta$ is infinitesimal real value.
\end{document}
