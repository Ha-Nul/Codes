\documentclass{article}
\usepackage[fleqn]{amsmath}
\usepackage{physics}
\usepackage{mathrsfs}
\usepackage{amssymb}
\usepackage{microtype}
\newcommand{\RN}[1]{%s
  \textup{\uppercase\expandafter{\romannumeral#1}}%
}
\usepackage{geometry}
\geometry{
 a4paper,
 left=20mm,
 right=20mm,
 top=15mm,
 bottom=20mm
 }
\begin{document}
\title{Gap problem}
\maketitle
1. Using sinh
\begin{align*}
    &\frac{e^{-\omega_k \tau}}{e^{\omega_k}-1}\\
    & = \frac{2i e^{-\omega_k \tau}}{2i e^{\frac{\omega_k}{2}} (e^{\frac{\omega_k}{2}}-e^{-\frac{\omega_k}{2}})} = \frac{e^{-\omega_k \tau - \frac{\omega_k}{2}}}{2i \sinh{\frac{\omega_k}{2}}}\\
    & ~ \frac{\frac{\omega_k}{2}}{\frac{\omega_k}{2}\sinh{\frac{\omega_k}{2}}}\quad , \quad \lim_{\omega \rightarrow 0} \frac{\frac{\omega_k}{2}}{\frac{\omega_k}{2}\sinh{\frac{\omega_k}{2}}} = \lim_{\omega \rightarrow 0} \frac{2}{\omega_k} \rightarrow \infty
\end{align*}
2. Change the Series Expansion form (Wolfram alpha)
\begin{align*}
    &\frac{e^{-\omega_k \tau}}{e^{\omega_k}-1}\\
    & ~ \frac{1}{\omega_k} - \tau - \frac{1}{2} + e^{-\omega_k\tau}\quad , \quad \lim_{\omega_k \rightarrow 0} \frac{1}{\omega_k} - \tau - \frac{1}{2} + e^{-\omega_k\tau} \rightarrow \infty
\end{align*}
\end{document}