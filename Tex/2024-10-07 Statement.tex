\documentclass[10pt,a4paper]{article}
% 필요한 패키지들
\usepackage[T1]{fontenc}
\usepackage{mathptmx}  % Times New Roman과 유사한 폰트
\usepackage[left=1.5cm,right=6.5cm,top=2.5cm,bottom=2.5cm,marginparwidth=5cm,marginparsep=0.5cm]{geometry}
\usepackage{fancyhdr}
\usepackage{lipsum}
\usepackage{marginnote}
\usepackage{titlesec}
\usepackage{xcolor}
\usepackage{titling}

% 색상 정의
\definecolor{sectionlinecolor}{RGB}{100,100,100} % 청회색
\begin{document}
\setcounter{page}{0}

% 페이지 스타일 설정
\pagestyle{fancy}
\fancyhf{}
\fancyfoot[R]{\thepage}  % 페이지 번호를 하단 우측으로 이동
\renewcommand{\headrulewidth}{0pt}
\renewcommand{\footrulewidth}{0pt}

% 제목 스타일 설정
\titleformat{\section}
  {\normalfont\Large\bfseries}{\thesection}{1em}{}[\titlerule[0pt]]
\titlespacing*{\section}{0pt}{3.5ex plus 1ex minus .2ex}{2.3ex plus .2ex}

\titleformat{\subsection}
  {\normalfont\large\bfseries}{\thesubsection}{1em}{}

  % 제목, 저자, 날짜 간격 조정
\setlength{\droptitle}{-60pt}
\pretitle{\begin{center}\Large\bfseries}
\posttitle{\par\end{center}\vskip 0.5em}
\preauthor{\begin{center}\large}
\postauthor{\par\end{center}}
\predate{\begin{center}\large}
\postdate{\par\end{center}}

% 문서 제목 정보
\title{Title}
\author{Hanul}
\date{\today}

% section 제목 아래 밑줄 추가
\newcommand{\sectionline}{%
  \nointerlineskip \vspace{\dimexpr-\parskip-2pt}%
  {\color{sectionlinecolor}\rule{\linewidth}{0.4pt}}%
  \vspace{\dimexpr-\parskip+3pt}%
}
\titleformat{\section}
  {\normalfont\Large\bfseries}
  {\thesection}
  {1em}
  {}
  [\sectionline]

% 문단 간격 설정
\setlength{\parskip}{6pt}

% 여백 노트 스타일 설정
\newcommand{\marginnotetext}[1]{%
  \marginnote{\raggedright\footnotesize#1}[1cm]%
}

% 제목 페이지 생성
\maketitle
\thispagestyle{fancy}  % 첫 페이지에도 페이지 번호 표시

\section{Introduction}
\lipsum[1]
\marginnotetext{This is a margin note for the introduction section. It should now be more to the left and not cut off.}

\subsection{Background}
\lipsum[2]
\marginnotetext{Additional information can be added here in the margin. This text should be fully visible.}

\lipsum[3]

\section{Methodology}
\lipsum[4-5]
\marginnotetext{You can use this space for comments, references, or additional explanations. The text should fit well now.}

\end{document}