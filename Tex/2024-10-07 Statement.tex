\documentclass[10pt,a4paper]{article}
% 필요한 패키지들
\usepackage[T1]{fontenc}
\usepackage{mathptmx}  % Times New Roman과 유사한 폰트
\usepackage[left=1.5cm,right=6.5cm,top=1.8cm,bottom=1.8cm,marginparwidth=5cm,marginparsep=0.5cm]{geometry}
\usepackage{fancyhdr}
\usepackage{lipsum}
\usepackage{marginnote}
\usepackage{titlesec}
\usepackage{xcolor}
\usepackage{titling}

% 색상 정의
\definecolor{sectionlinecolor}{RGB}{0, 132, 255} % 청회색
\begin{document}
\setcounter{page}{0}

% 페이지 스타일 설정
\pagestyle{fancy}
\fancyhf{}
\fancyfoot[R]{\thepage}  % 페이지 번호를 하단 우측으로 이동
\renewcommand{\headrulewidth}{0pt}
\renewcommand{\footrulewidth}{0pt}

% 제목 스타일 설정
\titleformat{\section}
  {\normalfont\Large\bfseries}{\thesection}{1em}{}[\titlerule[0pt]]
\titlespacing*{\section}{0pt}{2.0ex plus 1ex minus .1ex}{0ex}

\titleformat{\subsection}
  {\normalfont\large\bfseries}{\thesubsection}{1em}{}

  % 제목, 저자, 날짜 간격 조정
\setlength{\droptitle}{0pt}
\pretitle{\begin{center}\Large\bfseries}
\posttitle{\par\end{center}\vskip 0.5em}
\preauthor{\begin{center}\large}
\postauthor{\par\end{center}}

% 문서 제목 정보
\title{Title}
\author{Hanul}
\date{}

% section 제목 아래 밑줄 추가
\newcommand{\sectionline}{%
  \nointerlineskip \vspace{\dimexpr-\parskip-2pt}%
  {\color{sectionlinecolor}\rule{\linewidth}{0.8pt}}%
  \vspace{\dimexpr-\parskip+3pt}%
}
\titleformat{\section}
  {\normalfont\Large\bfseries}
  {\thesection}
  {1em}
  {}
  [\sectionline]

% 문단 간격 설정
\setlength{\parskip}{6pt}

% 여백 노트 스타일 설정
\newcommand{\marginnotetext}[1]{%
  \marginnote{\raggedright\footnotesize#1}[1cm]%
}

% 제목 페이지 생성
\maketitle
\thispagestyle{fancy}  % 첫 페이지에도 페이지 번호 표시

\section*{Dear Professor Gabriel T. Landi}
\begin{spacing}{1.5}
I am writing to briefly introduce myself as a prospective Ph.D. student in your laboratory. I hope my story will help you evaluate my potential as a candidate.

I first encountered the concept of quantum information when I was a child. Back then, I enjoyed reading several physics books, and one of them introduced me to the idea of information as a physical concept, which attracted me strongly.

I've graduated in physics at the Catholic University of Korea, a small yet delightful place. Since most professors majored in statistical physics, I've got a thorough basis of statistical mechanics and related theoretical frameworks, where I achieved a high score.

During my undergraduate studies, I became aware of the research field of quantum optics. I came across a paper and a book that discussed the concept of coherence and the general properties of photons. Although I recognized the various research areas within quantum optics, I was still uncertain about my specific research direction.

I am currently conducting research at the Daegu Gyeongbuk Institute of Science and Technology (DGIST) as a master's degree candidate. I met my current PI by chance and became involved in research related to open quantum systems. Through this research and an information theory course that I took during the first semester of my second year of graduate school, I was able to clarify my research interests.

My master's thesis is related to the Schmid quantum dissipative phase transition. In brief, I treated the resistively shunted Josephson junction as a physical simulator for the phase transition and mapped it onto a many-body problem, observing some remarkable results. To evaluate its dynamics in thermal dependency, I utilized a diagrammatic expansion method, which expands the system's dynamics perturbatively within the grand canonical ensemble framework. This method maps each term from the partition function as a series expansion onto featured topological structures, allowing for the exact calculation of all terms.

Focused on quantum optics and information for a long time, I am particularly interested in how systems lose their energy, or rather, information, through interaction with the environment over time. My expectation is to apply the methodology I learned during my master's degree to stochastic approaches and adapt it to open quantum systems.
\end{spacing}
\marginnotetext[15cm]{This is a margin note for the introduction section. It should now be more to the left and not cut off.}

\end{document}